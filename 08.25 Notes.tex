\documentclass{article}
\usepackage[utf8]{inputenc}

\title{08.25.2022 Notes}
\author{Math 403/503 }

\begin{document}

\maketitle

From last class... An alternative definition of subspace: A subset $U$ of a vector $V$ is a subspace if and only if $U$ is a vector space in it's own right. 

In set theory one of the first operations you learn is the union. Unfortuantely we can't use unions for subspaces because the union of two subspaces is usually not a subspace. \\

\textbf{Example}
Consider $V=R^2$. Given two vectors, $U_1$ and $U_2$ that are lines through the grid, the sum of those two vectors does not exists in the union. So, instead we look at the sums of vector spaces. \\

Definition: If $U_1$ and $U_2$ are subspaces of a vector space $V$ then $U_1 + U_2$ is the \underline{sum set} \{ $u_1 + u_2 | u_1 \epsilon U, u_2 \epsilon U$ \}. \\

\textbf{Example}: If $V = R^2$, $U_1 =$ the x- axis, and $U_2 =$ the y-axis then $U_1 + U_2 = R^2$. This means that the union of $U_1$ and $U_2$ is the x and y axes and when we add the two sets we get the coordinate plane. \\ \\
Rigorous Proof: Given any vector $(x,y)$ in $R^2$ we can write $(x,y) = (x,0) + (0,y)$. E.g. $(3,5) = (3,0) + (0,5)$. \\ \\ 

\textbf{U-Pruv}: In $R^4$ consider the subspaces: 
\begin{itemize}
    \item $U = \{(x, x, y, y) | x, y \epsilon R\}$
    \item $V = \{(x, x, x, y) | x,y \epsilon R \}$
    \item $U + V = \{(x, x, y, z) | x, y, z \epsilon R \}$
\end{itemize}

An alternative definition of the sum set...Let $V$ be a vector space and $U_1$, $U_2$ be subspaces of $V$. Then $U_1 + U_2$ is precisely the smallest subspace of $V$ containing both $U_1$ and $U_2$.\\

We observe there are two kinds of sums. (1) Unique: x-axis + y-axis = $R^2$. E.g. $(3,5) = (3, 0) + (0,5)$. There is no other way to write this! \textbf{This means the sum is DIRECT}. (2) Non-unique: xy-plane + yz-plane = $R^3$. E.g. $(1, 2, 3) = (1, 1, 0) + (0, 1, 3) = (1, 2, 0) + (0, 0, 3)$. Since there are multiple ways to write it, this makes the sum not direct. \\

Definition: When a sum $U_1 + U_2 = V$ has the property that any $v \epsilon V$ has a unique representation $v = u_1 + u_2$ where $u_1, u_2 \epsilon U$ we say the sum is \underline{direct} and write $U_1 \oplus U_2 = V$. \\

Lemma: A summation $U_1 + U_2$ of subspaces is direct if and only if \underline{0} = \underline{0} + \underline{0} cannot be written in any other way as a sum of elements of $U_1$ and $U_2$.\\\\
\textbf{\underline{Proof:}}\\ \\ $(\rightarrow)$ If the uniqueness property is true of any $v$ then in particular it is true of \underline{0}. \\
$(\leftarrow)$ Suppose the uniqueness property is not true of any vector $v$ (proof by contrapositive) then $v = u_1 + u_2$ where $u_1 \epsilon U, u_2 \epsilon U$ and $v = u_1' + u_2'$ where $u_1' \epsilon U, u_2' \epsilon U$. Note, $u_1 \neq u_1'$ and $u_2 \neq u_2'$. When we take the difference of these two equations, we get 
$v - v = u_1 + u_2 - u_1' - u_2'$. This gives us \underline{0} $= (u_1 - u_1') + (u_2 - u_2')$. Thus, we wrote \underline{0} as the sum of two nonzero vectors so the uniqueness property failed per \underline{0}.\\

The uniqueness property in direct sums should remind you of linear dependence: A sum $U_1 + U_2 + U_3$ is direct $(U_1 \oplus U_2 \oplus U_3)$ if and only if with any three vectors $u_1, u_2, u_3$, with $u_1 \epsilon U_1, u_2 \epsilon U_2, u_3 \epsilon U_3$ is linearly independent


\end{document}
