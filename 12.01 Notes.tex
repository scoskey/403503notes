\documentclass{article}
\usepackage[utf8]{inputenc}
\usepackage{amsmath}

\title{12.01 Notes}
\author{Math 403/503}
\date{December 2022}

\begin{document}

\maketitle

\section{Our Schedule}
Today - last lecture on determinants \\
Tuesday - review and homework questions \\
Wednesday - quiz is released (same format as the last two) \\
Thursday - optional class day; bring catch up questions, bring general questions, bring clarification questions about the quiz \\
Tuesday - quiz is due at the end of the day 

\section{Today's Lecture - Determinants}
We now define determinants and explore their properties analogously to our discussion of trace. \\\\
\textbf{Definition}: If $T \epsilon L(C^n)$ and $\lambda_1, ..., \lambda_n$ are the eigenvalues of $T$ with repetitions included, we let det(T) = $\lambda_1 * ... * \lambda_n$. \\\\
det(T) $= 0 \leftrightarrow T$ has 0 as an eigenvalue \\
$\leftrightarrow T$ is not invertible \\\\
We want to calculate det(T) from the matrix of $T$ (like trace). If $T$ has matrix $A$ in some basis and $A$ is upper triangular then det(T) = the product of the diagonal entries $a_{11}*...*a_{nn}$. \\\\
Unlike the trace, the diagonal product is NOT invariant under change of basis. So to find the determinant for a basis that doesn't make $T$ triangular, there is an extra step... Gauss - Jordan elimination. \\\\
Temporarily define the matrix determinant "n-det". \\\\
\textbf{Definition}: Let $A$ be an n x n matrix. We define... 
\begin{itemize}
    \item If $A$ is triangular, mdet(A) = $a_{11} * ... * a_{nn}$
    \item Adding a multiple of one row to another row does not change its mdet 
    \item Swapping any two rows negates mdet(A)
\end{itemize}
This means given any matrix we can calculate mdet(A) by first eliminating to upper triangular (keeping careful count of row exchanges) and then taking the diagonal. \\\\
Our goal is to show mdet is really equal to det. We mainly need that mdet doesn't depend on change of basis. \\\\
\textbf{Key Lemma}: mdet(AB) = mdet(A) mdet(B). \\\\
\textbf{Proof Sketch}: We start with 3 special cases when $A = E$ is an "elimination matrix": 
\begin{itemize}
    \item $E$ looks like the identity but has a single nonzero off diagonal entry. Then mdet(E) = 1 and mdet(EB) = mdet(B) because E just adds a multiple of one row to another. Thus, mdet(EB) = mdet(E) mdet(B). So we are happy in this case! 
    \item $E$ looks like an identity but with two rows exchanged. Then mdet(E) = -1 and mdet(EB) = - mdet(B) (because E swaps two rows of B). Thus, mdet(EB) = mdet(E) mdet(B)
    \item $E$ is the identity but with one of the one's replaced with another number, x. Then mdet(E) = x and mdet(EB) = x mdet(B). Thus, mdet(EB) = mdet(E) mdet(B). 
\end{itemize}
We now use the fact that any matrix $A$ is a product of elimination matrices. $A = E_1 * E_2 * ... * E_n$. Thus, mdet(AB) = mdet($E_1 * E_2 * ... * E_kB$) = mdet$(E_1)$ mdet $(E_2 ... E_k B)$ = ... = ... = mdet($E_1 ... E_k$) mdet(B) = mdet(A) mdet(B). \\\\
A consequence: mdet($B^{-1}AB$) = mdet($B^{-1}$) mdet(A) mdet(B) = mdet$(B)^ {-1}$ mdet(A) mdet(B) = mdet(A). \\\\
So mdet is basis independent. \\\\
\textbf{Theorem}: If $T$ has matrix $A$ then det(T) = mdet(A). \\
\textbf{Proof}: It's true if $A$ is triangular, otheriwse change basis to make it triangular. \\\\
From now on, we say "det" for both operators and matrices. \\\\
\textbf{Corollary}: det(ST) = det(S) det(T) because its true for trace. \\\\
The determinant has many applications, but one notable application to volumes slash calculus. Observation: Assume $T \epsilon L(R^n)$ and the matrix  of $T$ in the standard basis is diagonal. What does $T$ do to the volume of the unit box? T(unit box) = a box with volume $|\lambda_1 ... \lambda_n| =$ $|det T|$. T(any box B) = a box with volume $|det T|$ * vol(B). T(any set S) = a set with volume $|det T|$ * vol(S). \\\\
Surprisingly, this happen for ANY $T$, not just $T$ with a diagonal matrix! \\\\ 
\textbf{Theorem}: If $S$ is a subset of $R^n$ and $T \epsilon L(R^n)$ then vol(T(S)) = |det(T)| vol(A). \\
\textbf{Proof Sketch}: First do it assuming $T$ is positive. In this case, there is an orthonormal basis of $R^n$ where the matrix of $T$ is diagonal, and we can argue as above. For a general $T$, the polar decomposition states that $T = QP$ where $Q$ is orthonormal and $P$ is positive. $Q$ being orthonormal doesn't change volume also |det Q| =1 and $P$ being positive changes volume by det P. Thus $T = QP$ changes volume by |det Q det P| = |det T|. \\\\
This is why a determinant appears in substitution forumals in multivariable integration theory: Suppose $\phi: R^n \rightarrow R^n$ is a change of variables function and its Jacobian derivative (the matrix of partials) is J. Then the integral of f can be written as the integral of f composed with $\phi$ multipled by |det J|. 

\end{document}
