\documentclass{article}
\usepackage[utf8]{inputenc}

\title{08.30 Lecture Notes}
\author{Math 403/503 }

\begin{document}

\maketitle

Recall: $F = R$ or $C$ (these are scalars).$V =$ a vector space over $F$. This means addition and scalar multiplication are supported and several axioms state that they behave as expected. \\ \\
Definition: Given vectors $v_1, v_2, ... v_m$ in $V$, a \underline{linear combination} is any expression $x_1v_1 + ... + x_mv_m$ where $x_i \epsilon F$. \\ \\
Definition: Given vectors $v_1, v_2, ... v_m$ in $V$, the \underline{span} of $v_1, ..., v_m$ is the set of all linear combinations of $v_1, ..., v_m$. Terminology (could be a verb): If the span of $v_1, ..., v_m$ equals all of $V$ we say that $v_1, ..., v_m$ \underline{spans} $V$.\\

E.G. In $F^3$ the vectors $v_1 = (1, 0, 0)$ $v_2 = (0, 1, 0)$. The span of of $v_1, v_2$ consists of the xy-plane because linear combinations have the form $x(1, 0, 0) + y(0, 1, 0)$ or $(x, y, 0)$. The vectors $v_1 = (1, 0, 0), v_2 = (0, 1, 0), v_3 = (0, 0, 1)$ span $F^3$.\\\\
Definition: The \underline{dimension} of $V$ is the smallest number of vectors $v_1, ..., v_m$ that spans $V$. If there is no such number (so no finite list of vectors spans $V$) then $V$ is said to be infinite dimensional. 

E.G. $F^3$ is finite dimensional because $(1, 0, 0), (0, 1, 0), (0, 0, 1)$ span it. $F^N$ is not finite dimensional. 

E.G. A \underline{polynomial} is an expression of the form $a_0 + a_1z + a_2z^2 + ... + a_mz^m$ where $a_0, a_1, ... \epsilon F$. The \underline{degree} of the polynomial is the highest power $n$ of $z$ such that the coefficient $a_m \neq 0$.

If $V =$ the space of polynomials of degree $\leq m$ then $V$ is finite dimensional. If $W = $ the space of all polynomials then W is infinite dimensional. To see this let $w_1, w_2, ..., w_k$ be any list of elements of $W$. Each $w_i$ has a degree, $m_i$. Let $m=$ the maximum of these degrees $m_i$. Then the polynomial $z^(m+1)$ cannot be written as a linear combination of $w_1, ..., w_k$. Thus, no list $w_1, ..., w_k$ spans $W$.\\

\textbf{Linear Independence} \\
Given any vectors $v_1, ..., v_m$ in some space $V$, the $0$ vector is always a linear combination: $0 = 0v_1 + 0v_2 + ... + 0v_m$. \\
Definition: A set of vectors $v_1, ..., v_m$ is called \underline{linearly independent} if the only way to write $0$ as a combination of $v_1, ..., v_m$ is with all coefficients being $0$. If there is more than one way, we say $v_1, ..., v_m$ is linearly dependent. \\

Lemma: A set of vectors $v_1, ..., v_m$ is linearly independent if and only if none of the vectors $v_j$ can be written as a combination of the rest. The proof of this statement is homework. 

E.G. In $F^2$ consider the vectors $v_1 = (1,2), v_2 = (2,3), v_3 = (3,7)$. This list is linearly dependent: $(0,0) = 5(1,2) + -1(2,3) + -1(3,7)$ . Alternatively this could be written as: $(1,2) = 1/5(2,3) + 1/5(3,7)$. \\

\end{document}
