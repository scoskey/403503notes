\documentclass{article}
\usepackage[utf8]{inputenc}

\title{09.13 Notes}
\author{Math 403/503 }
\date{September 2022}

\begin{document}

\maketitle

\section{Null Spaces, Matrices}

We know about vector spaces. We know about linear transformations. Now we look at vector subspaces that are determined from linear transformations. \\\\
Definition: If $T \epsilon L(V,V)$, then the \underline{null space} (or \underline{kernel}) is: \\\\
{null T = \textbraceleft $v \epsilon V | T(v) = 0$\textbraceright} \\\\
Examples: 
\begin{itemize}
\item If T = 0 the 0 transformation then null T = V (the whole domain)
\item $T \epsilon L(V,V)$ is the identity $T(v) = v$, null T = \textbraceleft 0\textbraceright
\item $T\epsilon L(P(R), P(R))$, T = the derivative, null T = the constant functions (a one dimensional space of R)

\end{itemize}
Lemma: If $T\epsilon L(V,W)$ then null T is a subspace of V. In particular it's a vector space. \\\\
\textbf{Proof}: 
\begin{itemize} 
\item 0 exists in null T: $T0 = 0$ - check! 
\item If $v_1, v_2 \epsilon$ null T then $v_1 + v_2 \epsilon $ T: $Tv_1 = 0, Tv_2 = 0$, so $T(v_1 + v_2) = T(v_1) + T(v_2) = 0 + 0 = 0$. So $v_1 + v_2 \epsilon T$ 
\item If $v \epsilon $ null T, then $\alpha v \epsilon $ T: $Tv = 0$ so $T(\alpha v) = \alpha T(v) = \alpha 0 = 0$. QED.

\end{itemize}
Definition: If $T \epsilon L(V,W)$ the \underline{range} of T is: range T = \textbraceleft $T(v) | v\epsilon V$\textbraceright \\\\
Example: 
\begin{itemize}
\item T = 0, the zero transformation - range T = \textbraceleft 0\textbraceright
\item $T \epsilon L(R^2, R^3)$ defined by $T(x,y) = (2x, 5y, x+y)$ - range T = some plane in $R^2$
\item $T \epsilon L(P(R), P(R))$, T = the derivative - range T = all of $P(R)$

\end{itemize}

Lemma: Let $T\epsilon L(V,W)$ then range T is a subspace of W. \\\\

\textbf{Proof}: 
\begin{itemize}
\item $0 \epsilon $ range T: $T0=0$ - check! 
\item If $w_1, w_2 \epsilon$ range T then $w_1 + w_2 \epsilon$ range T: $w_1 = Tv_1$, $w_2 = Tv_2$ $\rightarrow$ $w_1 + w_2 = Tv_1 + Tv_2 = T(v_1 + v_2)$ so $w_1 + w_2 \epsilon $ range T.
\item If $w \epsilon $ range T and $\alpha \epsilon F$ then $\alpha w \epsilon $ range T: $w = Tv \rightarrow \alpha w = \alpha Tv = T(\alpha v)$. So $\alpha w$ is in range T. QED.
\end{itemize}
Recall if $f: X \rightarrow Y $ then function f is \underline{injective} means: $x_1 \neq x_2 \rightarrow f(x_1) \neq f(x_2)$. \\
The function f is \underline{surjective} onto Y means: for every $y \epsilon Y$ there exists an $x \epsilon X$ such that $f(x) = y$. \\\\
Lemma: a linear transformation $T \epsilon L(V,W)$ is injective IFF null T = \textbraceleft 0\textbraceright \\\\
\textbf{Proof}: ($\rightarrow$) Suppose T is injective. Recall null(T) = everything that maps to 0. Since T is injective, at most one point in V can map to 0. We know $T0 = 0$ and now there can be nothing else! So null T = \textbraceleft 0\textbraceright.\\
$(\leftarrow)$ Suppose null T = \textbraceleft 0\textbraceright. Suppose $Tv_1 = Tv_2$. We want to show $v_1 = v_2$ (this is injective in its contrapositive form). Then $Tv_1 - Tv_2 = 0$, $T(v_1 - v_2) = 0$, $v_1 - v_2 \epsilon $ null T, $v_1 - v_2 = 0$, $v_1 = v_2$. QED. 


\end{document}
